
% Plantilla para presentaci�n de art�culos
% VII Seminario Internacional de Procesamiento y An�lisis de Im�genes M�dicas
% SIPAIM 2011

% Plantilla de art�culo para la clase LaTeX 'elsarticleSIPAIM',
% basada en la clase LaTeX para art�culos de Elsevier
% 'elsarticle' Copyright 2007, 2008, 2009 Elsevier Ltd
% y distribu�da bajo la licencia LaTeX Project Public License

% Opciones para los art�culos de SIPAIM 2011:
% - Tama�o de papel: carta
% - Tama�o de fuente: 11pt
% - Distribuci�n del texto: una sola columna
% IMPORTANTE: NO MODIFICAR ESTAS OPCIONES (el art�culo puede ser rechazado por no cumplir con el estilo)
\documentclass[letterpaper,11pt,onecolumn]{elsarticleSIPAIM}

% Inclusi�n de paquetes necesarios
% paquetes para funciones y s�mbolos matem�ticos
\usepackage{amsmath,amssymb}
% paquete para el tipo de codificaci�n del texto
\usepackage[latin1]{inputenc}
% paquete para la subdivisi�n de palabras y opciones del lenguaje, opci�n spanish para espa�ol
\usepackage[spanish]{babel}
% paquetes para inclusi�n de figuras y subfiguras
\usepackage{graphicx,subfigure}
% paquete para formato de direcciones de p�ginas web y correos electr�nicos
\usepackage{url}


% inicio del documento
\begin{document}

% cabecera del documento
% incluye: t�tulo, autores, afiliaciones, resumen y palabras clave
\begin{frontmatter}
  
  % t�tulo
  \title{T�tulo}

  % autores y afiliaciones
  % utilizar etiquetas para relacionar cada autor a una o varias afiliaciones
  % utilizar \corref[<etiqueta>] para indicar el autor de correspondencia
  \author[ent1,ent2]{Autor 1\corref{cor1}}
  \author[ent1]{Autor 2}
  \author[ent2]{Autor 3}
  
  % cada afiliaci�n debe inclu�r: grupo, laboratorio y/o universidad, ciudad, pa�s
  \address[ent1]{Afiliaci�n 1}
  \address[ent2]{Afiliaci�n 2}
  
  % texto del autor de correspondencia, debe inclu�r: direcci�n f�sica, tel�fono y correo electr�nico
  % utilizar \url{<direccion>} para el correo electr�nico
  \cortext[cor1]{Correspondencia: Autor 1. Direcci�n, tel�fono, \url{correo@electronico}}
  
  \begin{abstract}
    % texto del resumen
  \end{abstract}
  
  \begin{keyword}
    % palabras clave, de la siguiente forma: palabra 1 \sep palabra 2 \sep palabra 3
  \end{keyword}
% fin de la cabecera
\end{frontmatter}


% cuerpo del documento
\section{}
\subsection{}


% referencias
% el estilo corresponde a: referencias numeradas
\bibliographystyle{elsarticle-num}
% indicar el nombre del archivo .bib (sin la extensi�n) que incluye las referencias del art�culo
% \bibliography{<archivo-referencias>}

% fin del documento
\end{document}
